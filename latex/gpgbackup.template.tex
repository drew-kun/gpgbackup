\documentclass{article}

\usepackage{float}
\usepackage{graphicx}
\usepackage{pgffor}
\usepackage{fancyhdr}

\pagestyle{fancy}
\fancyhf{}
\fancyhead[LE,RO]{\LaTeX}
\fancyhead[RE,LO]{PRIVATE_KEY.asc}
\fancyfoot[CE,CO]{\leftmark}
\fancyfoot[C]{\thepage \ of 2}

\renewcommand{\headrulewidth}{1pt}
\renewcommand{\footrulewidth}{1pt}

\def\gpgkey{PRIVATE_KEY}

\newcommand\gpgtopng{
  \immediate\write18{GPG --export-secret-key \gpgkey | paperkey --output-type raw | base64 > \gpgkey.ascii}
  \immediate\write18{SPLIT -n 4 -d \gpgkey.ascii qr_}
  \immediate\write18{for f in qr_*; do cat $f | qrencode -o /tmp/${f}.png; done}
  \immediate\write18{SECRM qr_* \gpgkey.ascii}
}

\gpgtopng

\begin{document}

\foreach \l in {0, ..., 3} {
  \begin{figure}[H]
    \centering
    \includegraphics[width=0.6\textwidth]{/tmp/qr_0\l.png}
    \caption{QR \l}
  \end{figure}
}

\centering{To recover the key run the following command:}
\begin{verbatim}
for f in *.png; do zbarimg --raw ${f} | HEAD -c -1 > ${f}.out ; done
cat *.out | base64 DECODE | paperkey --pubring ~/.gnupg/pubring.gpg | GPG --import
\end{verbatim}

\end{document}
